COREAD_CCLE<-read.table(gzfile("COREAD_CCLE.txt.gz"),header=TRUE,sep="\t")
CCLE_matrix<-as.matrix(COREAD_CCLE[,1:40])

COREAD_proteome<-read.table("COREAD_proteome_43lines.txt", sep="\t",header=TRUE)
proteome_matrix<-as.matrix(COREAD_proteome[,1:43])
dimnames(proteome_matrix)[[1]]<-COREAD_proteome$Gene.ID
NAN<-rowSums(is.na(proteome_matrix))
table(NAN)
pep<-which(NAN==0)
################### all protein presented ###############
proteome_NANreduced<-proteome_matrix[pep,]


#Data_RNA_total<-read.table(gzfile("Data_RNA_total.txt.gz"),header=TRUE,sep="\t")
#Data_RNA_transformed<-read.table(gzfile("Data_RNA_transformed.txt"),header=TRUE,sep="\t")

########################### Gene Protein correlation #######################
                            make the same cell lines and genes for CCLE and proteins
############################################################################
str(proteome_NANreduced)
#match_cell_line<-match(dimnames(CCLE_matrix)[[2]],dimnames(proteome_NANreduced)[[2]])
proteome_NANreduced_CCLE<-proteome_NANreduced[,1:40]
match_gene<-match(dimnames(proteome_NANreduced_CCLE)[[1]],dimnames(CCLE_matrix)[[1]])
CCLE_proteome<-CCLE_matrix[match_gene,]

str(proteome_NANreduced_CCLE)
str(CCLE_proteome)
##### nan produced remove non reliable genes 
zeros<-rowSums(CCLE_proteome<=1)
table(zeros)
k<-which(zeros==0)
CCLE_proteome<-CCLE_proteome[k,]
proteome_NANreduced_CCLE<-proteome_NANreduced_CCLE[k,]

str(proteome_NANreduced_CCLE)
str(CCLE_proteome)
############### make CCLE and proteome  comparable ############

str(CCLE_proteome)
Row_mean<-rowMeans(CCLE_proteome)
Row_mean_extracted<-(CCLE_proteome/Row_mean)
multiplication<-Row_mean_extracted*100


 
 str(proteome_NANreduced_CCLE)
 str(multiplication)
 
 Transcriptome<-multiplication
 Proteome<-proteome_NANreduced_CCLE
 
 str(Transcriptome)
 str(Proteome)
 
 ####################################### remove mitochondira genes  ######################
#########################################

mitocondria<-c("ENSG00000210049", "ENSG00000211459", "ENSG00000210077", "ENSG00000210082", 
"ENSG00000209082", "ENSG00000198888", "ENSG00000210100", "ENSG00000210107", 
"ENSG00000210112", "ENSG00000198763", "ENSG00000210117", "ENSG00000210127", 
"ENSG00000210135", "ENSG00000210140", "ENSG00000210144", "ENSG00000198804", 
"ENSG00000210151", "ENSG00000210154", "ENSG00000198712", "ENSG00000210156", 
"ENSG00000228253", "ENSG00000198899", "ENSG00000198938", "ENSG00000210164", 
"ENSG00000198840", "ENSG00000210174", "ENSG00000212907", "ENSG00000198886", 
"ENSG00000210176", "ENSG00000210184", "ENSG00000210191", "ENSG00000198786", 
"ENSG00000198695", "ENSG00000210194", "ENSG00000198727", "ENSG00000210195", 
"ENSG00000210196")


mit_match<-match(mitocondria,dimnames(Transcriptome)[[1]])
d<-which(!is.na(mit_match))
Transcriptome<-Transcriptome[-mit_match[d],]
Proteome<-Proteome[-mit_match[d],]
mit_match<-match(mitocondria,dimnames(Proteome)[[1]])

 Go<-read.table("GO_0044455.txt",sep=",",header=TRUE)  #####mitochondrial membrane part
Gene.stable.ID<-unique(Go$Gene.stable.ID)

match_Go<-match(Gene.stable.ID,dimnames( Transcriptome)[[1]])
Na<-which(!is.na(match_Go))
Transcriptome <-Transcriptome[-match_Go[Na], ]
Proteome <-Proteome[-match_Go[Na], ]



################################# removing ribosomes######################

Go_mit_rib<-read.table("GO_0005840.txt",sep=",",header=TRUE)
Gene.stable.ID.mit_rib<-unique(Go_mit_rib$Gene.stable.ID)
match_Go_mit_rib<-match(Gene.stable.ID.mit_rib,dimnames( Transcriptome)[[1]])
Na_mit_rib<-which(!is.na(match_Go_mit_rib))
Transcriptome <-Transcriptome[-match_Go_mit_rib[Na_mit_rib], ]
Proteome <-Proteome[-match_Go_mit_rib[Na_mit_rib], ]

 str(Transcriptome)
 str(Proteome)
 ###################################### find non unique values
 
 str(unique(dimnames(Proteome)[[1]]))
str(unique(dimnames(Transcriptome)[[1]]))
duplicated_proteome<-duplicated(dimnames(Proteome)[[1]])
D_proteome<-which(duplicated_proteome=="TRUE")


duplicated_Transcriptome<-duplicated(dimnames(Transcriptome)[[1]])
D_Transcriptome<-which(duplicated_Transcriptome=="TRUE")

Proteome<-Proteome[-D_proteome,]
Transcriptome<-Transcriptome[-D_Transcriptome,]

 str(Transcriptome)
 str(Proteome)
 
 
 
d<-length(dimnames(Proteome)[[2]])

corr_cell<-matrix(0,nrow=d,ncol=d)
for(i in 1:d){
 for(j in 1:d){

corr_cell[i,j]<-cor(Proteome[,i],Transcriptome[,j],method="spearman")

 } 
 }
dimnames(corr_cell)[[1]]<-dimnames(Proteome)[[2]]
 
 dimnames(corr_cell)[[2]]<-dimnames(Proteome)[[2]]

 library(gplots)
 heatmap.2(corr_cell,Rowv=NA,Colv=NA,scale="none",col=bluered(256),key=TRUE,symkey=FALSE, density.info="none",trace="none",margins=c(6,6),main="Cell line specific Spearman correlation ",cexRow=0.75,cexCol=0.75,)

####################### find variable proteins

sd_Proteome<-apply(Proteome, 1, sd)
median_sd_Proteome<-median(sd_Proteome)
Normalized_Proteome_sd<-sd_Proteome/median_sd_Proteome 
 

####################################################################
####################################################

#install.packages("r.jive")
library(r.jive)
 Data_test<- list(Proteome,Transcriptome)
 names(Data_test)<-c("Proteome","Transcriptome")
 Results = jive(Data_test)
 
 str(Results)
 
#heatmap((Results$joint)[[1]])
#heatmap(Results$data$Proteome)

str(Results$joint[[1]])


################## visualization ################### 
png("VarExplained.png",height=300,width=450)  
showVarExplained(Results)  
dev.off()


png("HeatmapsBRCA.png",height=465,width=705)
showHeatmaps(Results, order_by=-1)  #### the matrices are not re-ordered  If  order_by=0, orderings are determined by joint structure
                                                                              
dev.off() 
######################################################
Specifies how to order the rows and columns of the heatmap.
          If order_by=-1, the matrices are not re-ordered.  If
          order_by=0, orderings are determined by joint structure.
          Otherwise, order_by gives the number of the individual
          structure dataset to determine the ordering. In all cases
          orderings are determined by complete-linkage hiearchichal
          clustering of Euclidean distances.

show_all: Specifies whether to show the full decomposition of the data,
          JIVE estimates, and noise.  If show_all=FALSE, only the
          matrix (or matrices) that determined the column ordering is
          shown.

library(gplots)
heatmap.2(Results$joint[[1]],dendrogram='none', Rowv=FALSE, Colv=FALSE,trace='none')



################## how to get the clustering of row and columns
   #############################################################
 order_by=0
old.par <- par(no.readonly = TRUE) # all par settings which could be changed
  on.exit(par(old.par))
  
  l <- length(Results$data)
 
  ####Get row/column orderings
  Mat_ColOrder <- do.call(rbind,Results$joint)
  row.orders = list()
  if(order_by==-1){
  	for(i in 1:l) row.orders[[i]]=c(dim(Results$data[[i]])[1]:1)
  	col.order <- c(1:dim(Results$data[[i]])[2])
  }
  if(order_by>-1){
  	if(order_by>0) {Mat_ColOrder = Results$individual[[order_by]]}
  	col.dist<-dist(t(Mat_ColOrder))
  	rm(Mat_ColOrder)
  	col.order<-hclust(col.dist)$order
  	for(i in 1:l){
    	if(order_by==0) {row.dist <- dist(Results$joint[[i]])}
    	else{row.dist <- dist(Results$individual[[i]])}
    	row.orders[[i]] <- hclust(row.dist)$order
  }}

plot(hclust(col.dist))
plot(hc,labels=dimneams(Proteome)[[2]],cex=0.5)

col_order_new<-dimnames(Proteome)[[2]][col.order]

  Image_Joint = list()
for(i in 1:l){ Image_Joint[[i]] = as.matrix(Results$joint[[i]][row.orders[[1]],col.order]) }
heatmap.2(Image_Joint[[1]],dendrogram='none', Rowv=FALSE, Colv=FALSE,trace='none', col=bluered(100),ColSideColors=subtypes)
subtypes<-as.character(subtypes)
heatmap.2(Image_Joint[[1]],dendrogram='none', col=bluered(100),ColSideColors=subtypes)

heatmap(Image_Joint[[1]], cluster_rows = FALSE, cluster_columns = FALSE)
dimnames(Image_Joint[[1]])[[2]]<-col_order_new
my_palette <- colorRampPalette(c("blue","white","red"))(n = 100)
heatmap(Image_Joint[[1]], Rowv = NA,col=my_palette)
heatmap(Image_Joint[[1]], Rowv = NA, Colv = NA,col=bluered(100),ColSideColors=subtypes[col.order])

############# this is the function in rjive
library(fields)
show.image = function(Image,ylab=''){
  lower = mean(Image)-3*sd(Image)
  upper = mean(Image)+3*sd(Image)
  Image[Image<lower] = lower
  Image[Image>upper] = upper
  image.plot(x=1:dim(Image)[2], y=1:dim(Image)[1], z=t(Image), zlim = c(lower,upper),axes=FALSE,col=bluered(100),xlab="",ylab=ylab, add=FALSE)
  axis(1, at=seq(1,40, length=40), labels=col_order_new,pos=0,lwd=0,las=2,cex.lab=0.1)
} 
show.image(Image_Joint[[2]])

##############################################

################ find the cell line types
library(readxl)
test<-read_excel("Driver_mut_vs_subtypes .xlsx")

colnames(test)<-gsub("-", "", colnames(test)) 
test.matrix<-as.matrix(test[,2:51])
common_lines<-match(dimnames(Proteome)[[2]],dimnames(test.matrix)[[2]] )
subtypes<- test.matrix[6,common_lines]
subtypes<-as.integer(subtypes)

attributes(subtypes)<-dimnames(Proteome)[[2]]

Colors = rep('black',40)
Colors[subtypes==2] = 'red'
Colors[subtypes==3] = 'green'
Colors[subtypes==4] = 'blue'
Colors[subtypes==5] = 'cyan1'

showPCA(Results,n_joint=2,Colors=Colors)
showPCA(Results,n_joint=2,n_indiv=c(2,2),Colors=Colors)


########PCA
plot(SVD$d^2/sum(SVD$d^2), xlim = c(0, 5), type = "b", pch = 16, xlab = "principal components", 
    ylab = "variance explained")
########### separate the pca in each proteome and transcriptome)
prot_PCA<-abs(SVD$u[1:length(dimnames(Proteome)[[1]]),1])
which(prot_PCA==max(prot_PCA), arr.ind=TRUE)
p<-head(sort(prot_PCA,decreasing=TRUE),10)
o<-match(p,prot_PCA)
jprot<-dimnames(Proteome)[[1]][o]


trans_PCA<-abs(SVD$u[5219:10436,2])
which(trans_PCA==max(trans_PCA), arr.ind=TRUE)
p<-head(sort(trans_PCA,decreasing=TRUE),40)
o<-match(p,trans_PCA)
jtrans<-dimnames(Transcriptome)[[1]][o]

intersect(jprot,jtrans)

################################### look for the intersection
biocLite("limma")
library(limma)
library(venn)
venn(degree)
install.packages('VennDiagram')
library(VennDiagram)
venn(degree)

draw.pairwise.venn(40, 40, 18, category = c("Transcriptome", "Proteome"), lty = rep("blank", 
    2), fill = c("light blue", "pink"), alpha = rep(0.5, 2), cat.pos = c(0, 
    0), cat.dist = rep(0.025, 2))

############################################################################### proteome


Top_PCA_prot<-head(sort(prot_PCA,decreasing=TRUE),40)
o<-match(Top_PCA_prot,prot_PCA)
Top_prot<-dimnames(Proteome)[[1]][o]


Down_PCA_prot<-tail(sort(prot_PCA,decreasing=TRUE), n=40)
o<-match(Down_PCA_prot,prot_PCA)
Down_prot<-dimnames(Proteome)[[1]][o]

PCA_prot <- data.frame(Ensemble=c(Top_prot, Down_prot))
PCA_prot$group <- "Down"
temp<-length(Top_prot)
PCA_prot$group[1:temp] <- "Top"
PCA_prot$group<-as.factor(PCA_prot$group)

library(clusterProfiler)

library(org.Hs.eg.db)
keytypes(org.Hs.eg.db)

xx.formula <- compareCluster(Ensemble~group, data=PCA_prot,universe = dimnames(Proteome)[[1]],
fun='enrichGO', OrgDb='org.Hs.eg.db',keyType = "ENSEMBL",ont="CC")
as.data.frame(xx.formula)

dotplot(xx.formula)

ego_Down <- enrichGO(gene          = Top_prot,
                universe      = dimnames(Proteome)[[1]],
                OrgDb         = org.Hs.eg.db,
                keyType = "ENSEMBL",
                ont           = "CC",
                pAdjustMethod = "BH",
                pvalueCutoff  = 0.01,
                qvalueCutoff  = 0.05,
        readable      = TRUE)
        
head(ego_Down)
 
dotplot(ego_Down)

######################  ###################### check the same for the joint transcriptome####################################
Top_PCA_trans<-head(sort(trans_PCA,decreasing=TRUE),40)
o<-match(Top_PCA_trans,trans_PCA)
Top_trans<-dimnames(Transcriptome)[[1]][o]

Down_PCA_trans<-tail(sort(trans_PCA,decreasing=TRUE), n=40)
o<-match(Down_PCA_trans,trans_PCA)
Down_trans<-dimnames(Transcriptome)[[1]][o]

PCA_trans <- data.frame(Ensemble=c(Top_trans, Down_trans))
PCA_trans$group <- "Down"
temp<-length(Top_trans)
PCA_trans$group[1:temp] <- "Top"
PCA_trans$group<-as.factor(PCA_trans$group)


xx.formula_trans <- compareCluster(Ensemble~group, data=PCA_trans,universe = dimnames(Transcriptome)[[1]],
fun='enrichGO', OrgDb='org.Hs.eg.db',keyType = "ENSEMBL",ont="CC")
as.data.frame(xx.formula_trans)

dotplot(xx.formula_trans)

ego_Down <- enrichGO(gene          = Top_trans,
                universe      = dimnames(Proteome)[[1]],
                OrgDb         = org.Hs.eg.db,
                keyType = "ENSEMBL",
                ont           = "CC",
                pAdjustMethod = "BH",
                pvalueCutoff  = 0.01,
                qvalueCutoff  = 0.05,
        readable      = TRUE)









str(SVD$u)
posi<-abs(SVD$u)
which(posi[,1]==max(posi[,1]), arr.ind=TRUE)
k<-rbind(Results$joint[[1]],Results$joint[[2]])
p<-head(sort(posi[,1], decreasing=TRUE),10)
o<-match(p,posi[,1])
total<-rbind(Proteome,Transcriptome)

dimnames(total)[[1]][o]


Image_Joint[[1]][1,]
Results$joint[[1]][1005,col.order]

highy_variation<-dimnames(Proteome)[[1]][row.orders[[1]][1:622]]


if(n_indiv[i]>0){
    SVD = svd(result$individual[[i]],nu=n_indiv[i],nv=n_indiv[i])
    indices = (n_joint+sum(n_indiv[0:(i-1)])+1):(n_joint+sum(n_indiv[0:i]))
    PCs[indices,] = diag(SVD$d)[1:n_indiv[i],1:n_indiv[i]]%*%t(SVD$v[,1:n_indiv[i]])
    PC_names[indices] = paste(names(result$data)[i]," Indiv ",1:n_indiv[i]) 


p<-head(sort(posi[,1],decreasing=TRUE),50)
oo<-match(p,posi[,1])
dimnames(total)[[1]][oo]



Top_mit_GO_rib<-head(Sort_corr,100)
Down_mit_GO_rib<-tail(Sort_corr, n=100)


mydf_mit_GO_rib <- data.frame(Ensemble=c(names(Top_mit_GO_rib),names(Down_mit_GO_rib)))
mydf_mit_GO_rib$group <- "Down"
temp<-length(names(Top_mit_GO_rib))
mydf_mit_GO_rib$group[1:temp] <- "Top"
mydf_mit_GO_rib$group<-as.factor(mydf_mit_GO_rib$group)

library(clusterProfiler)

library(org.Hs.eg.db)
keytypes(org.Hs.eg.db)

xx.formula_mit_GO_rib <- compareCluster(Ensemble~group, data=mydf_mit_GO_rib,universe = names(Sort_corr),
fun='enrichGO', OrgDb='org.Hs.eg.db',keyType = "ENSEMBL",ont="CC")
as.data.frame(xx.formula_mit_GO_rib)
dotplot(xx.formula_mit_GO_rib,showCategory=15)
dotplot(xx.formula_mit_GO_rib)
###################################### PCA related to the bsulute value

str(SVD$u)
ABS<-abs(SVD$u[,1])
which(ABS==max(ABS), arr.ind=TRUE)
k<-c(corr,corr)
p<-tail(sort(ABS),20)
o<-match(p,ABS)

pp<-head(sort(ABS),20)
oo<-match(pp,ABS)


total<-rbind(Proteome,Transcriptome)

Top_mit_GO_rib<-dimnames(total)[[1]][o]
Down_mit_GO_rib<-dimnames(total)[[1]][oo]

mydf_mit_GO_rib <- data.frame(Ensemble=c(Top_mit_GO_rib,Down_mit_GO_rib))
mydf_mit_GO_rib$group <- "Down"
temp<-length(Top_mit_GO_rib)
mydf_mit_GO_rib$group[1:temp] <- "Top"
mydf_mit_GO_rib$group<-as.factor(mydf_mit_GO_rib$group)

library(clusterProfiler)

library(org.Hs.eg.db)
keytypes(org.Hs.eg.db)

CC <- enrichGO(gene          = Top_mit_GO_rib,
                universe      = dimnames(Proteome)[[1]],
                OrgDb         = org.Hs.eg.db,
                keyType = "ENSEMBL",
                ont           = "CC",
                pAdjustMethod = "BH",
                pvalueCutoff  = 0.01,
                qvalueCutoff  = 0.05,
        readable      = TRUE)

dotplot(CC)



xx.formula_mit_GO_rib <- compareCluster(Ensemble~group, data=mydf_mit_GO_rib,universe = dimnames(Proteome)[[1]],
fun='enrichGO', OrgDb='org.Hs.eg.db',keyType = "ENSEMBL",ont="CC")
as.data.frame(xx.formula_mit_GO_rib)
dotplot(xx.formula_mit_GO_rib,showCategory=15)
dotplot(xx.formula_mit_GO_rib)

################ PCA related to individual proteome

SVD_ind = svd(result$individual[[1]])
ABS_ind<-abs(SVD_ind$u[,1])
which(ABS_ind==max(ABS_ind), arr.ind=TRUE)
k<-Results$data$Proteome
p<-tail(sort(ABS_ind),10)
o<-match(p,ABS_ind)
pp<-head(sort(ABS_ind),10)
oo<-match(pp,ABS_ind)
Top_mit_GO_rib<-dimnames(Proteome)[[1]][o]
Down_mit_GO_rib<-dimnames(Proteome)[[1]][oo]

mydf_mit_GO_rib <- data.frame(Ensemble=c(Top_mit_GO_rib,Down_mit_GO_rib))
mydf_mit_GO_rib$group <- "Down"
temp<-length(Top_mit_GO_rib)
mydf_mit_GO_rib$group[1:temp] <- "Top"
mydf_mit_GO_rib$group<-as.factor(mydf_mit_GO_rib$group)

xx.formula_mit_GO_rib <- compareCluster(Ensemble~group, data=mydf_mit_GO_rib, universe = dimnames(Proteome)[[1]],
fun='enrichGO', OrgDb='org.Hs.eg.db',keyType = "ENSEMBL",ont="CC")
as.data.frame(xx.formula_mit_GO_rib)
dotplot(xx.formula_mit_GO_rib,showCategory=15)
dotplot(xx.formula_mit_GO_rib)


################ PCA related to individual transcriptome

SVD_ind_trans = svd(result$individual[[2]])
ABS_ind_trans<-abs(SVD_ind_trans$u[,2])
which(ABS_ind_trans==max(ABS_ind_trans), arr.ind=TRUE)
p_trans<-tail(sort(ABS_ind_trans),100)
o_trans<-match(p_trans,ABS_ind_trans)
pp_trans<-head(sort(ABS_ind_trans),100)
oo_trans<-match(pp_trans,ABS_ind_trans)
Top_mit_GO_rib_trans<-dimnames(Transcriptome)[[1]][o_trans]
Down_mit_GO_rib_trans<-dimnames(Transcriptome)[[1]][oo_trans]

mydf_mit_GO_rib_trans <- data.frame(Ensemble=c(Top_mit_GO_rib_trans,Down_mit_GO_rib_trans))
mydf_mit_GO_rib_trans$group <- "Down"
temp<-length(Top_mit_GO_rib_trans)
mydf_mit_GO_rib_trans$group[1:temp] <- "Top"
mydf_mit_GO_rib_trans$group<-as.factor(mydf_mit_GO_rib_trans$group)

xx.formula_mit_GO_rib_trans <- compareCluster(Ensemble~group, data=mydf_mit_GO_rib_trans, universe = dimnames(Transcriptome)[[1]],
fun='enrichGO', OrgDb='org.Hs.eg.db',keyType = "ENSEMBL", ont="BP)

as.data.frame(xx.formula_mit_GO_rib_trans)
dotplot(xx.formula_mit_GO_rib,showCategory=15)
dotplot(xx.formula_mit_GO_rib_trans)










##################    ################################## now make the correlation after remoing all of them######################

d<-length(dimnames(Proteome)[[1]])

corr<-c(rep(0,d))
for(i in 1:d){

 corr[i]<-cor(Proteome[i,],Transcriptome[i,],method="spearman")

 } 
 names(corr)<-dimnames(Proteome)[[1]]

 Sort_corr<-sort(corr, decreasing = TRUE)

library(ggplot2)
Correlation<- Sort_corr
index <- Sort_corr
Gene<-c(1:length(corr))

qplot(Gene,Correlation, colour=index) + scale_colour_gradient(limits=c(-0.5, 1),low="blue", high="red")+ggtitle("Gene / Protein Spearman Correlation")+theme(plot.title = element_text(hjust = 0.5))+ ylim(-0.5,1)
######################################################### lets see the GO
Top_mit_GO_rib<-head(Sort_corr,500)
Down_mit_GO_rib<-tail(Sort_corr, n=500)


mydf_mit_GO_rib <- data.frame(Ensemble=c(names(Top_mit_GO_rib),names(Down_mit_GO_rib)))
mydf_mit_GO_rib$group <- "Down"
temp<-length(names(Top_mit_GO_rib))
mydf_mit_GO_rib$group[1:temp] <- "Top"
mydf_mit_GO_rib$group<-as.factor(mydf_mit_GO_rib$group)

library(clusterProfiler)

library(org.Hs.eg.db)
keytypes(org.Hs.eg.db)

xx.formula_mit_GO_rib <- compareCluster(Ensemble~group, data=mydf_mit_GO_rib,universe = names(Sort_corr),
fun='enrichGO', OrgDb='org.Hs.eg.db',keyType = "ENSEMBL",ont="CC")
as.data.frame(xx.formula_mit_GO_rib)
dotplot(xx.formula_mit_GO_rib,showCategory=15)
dotplot(xx.formula_mit_GO_rib)

p<-which(names(corr_GDSC_mit_Go_Rib)=="ENSG00000134480")  #### chromosomal region######
plot(GDSC_mit_Go_rib[p,],proteomics_matrix_mit_Go_rib[p,],xlab="mRNA",ylab="proteomics")
xx.formula_mit_GO_rib@compareClusterResult[1,]

p<-which(names(corr_GDSC_mit_Go_Rib)=="ENSG00000051180")  #### chromosomal region######

plot(GDSC_mit_Go_rib[p,],proteomics_matrix_mit_Go_rib[p,],xlab="mRNA",ylab="proteomics",main="RBMS1, rhh=0.93" )














################################ Multiple Co-inertia


COREAD_Genetech<-read.table(gzfile("COREAD_Genetech.txt.gz"),header=TRUE,sep="\t")
Genetech_matrix<-as.matrix(COREAD_Genetech[,1:28])

COREAD_CCLE<-read.table(gzfile("COREAD_CCLE.txt.gz"),header=TRUE,sep="\t")
CCLE_matrix<-as.matrix(COREAD_CCLE[,1:40])

COREAD_proteome<-read.table("COREAD_proteome_43lines.txt", sep="\t",header=TRUE)
proteome_matrix<-as.matrix(COREAD_proteome[,1:43])
dimnames(proteome_matrix)[[1]]<-COREAD_proteome$Gene.ID
NAN<-rowSums(is.na(proteome_matrix))
table(NAN)
pep<-which(NAN==0)
################### all protein presented ###############
proteome_NANreduced<-proteome_matrix[pep,]


#Data_RNA_total<-read.table(gzfile("Data_RNA_total.txt.gz"),header=TRUE,sep="\t")
#Data_RNA_transformed<-read.table(gzfile("Data_RNA_transformed.txt"),header=TRUE,sep="\t")

########################### Gene Protein correlation #######################
                            make the same cell lines and genes for CCLE and proteins
############################################################################
str(proteome_NANreduced)
#match_cell_line<-match(dimnames(CCLE_matrix)[[2]],dimnames(proteome_NANreduced)[[2]])
proteome_NANreduced_CCLE<-proteome_NANreduced[,1:40]
match_gene<-match(dimnames(proteome_NANreduced_CCLE)[[1]],dimnames(CCLE_matrix)[[1]])
CCLE_proteome<-CCLE_matrix[match_gene,]

str(proteome_NANreduced_CCLE)
str(CCLE_proteome)
##### nan produced remove non reliable genes 
zeros<-rowSums(CCLE_proteome==0)
table(zeros)
k<-which(zeros==0)
CCLE_proteome<-CCLE_proteome[k,]
proteome_NANreduced_CCLE<-proteome_NANreduced_CCLE[k,]

str(proteome_NANreduced_CCLE)
str(CCLE_proteome)

str(CCLE_proteome)
Row_mean<-rowMeans(CCLE_proteome)
Row_mean_extracted<-(CCLE_proteome/Row_mean)
multiplication<-Row_mean_extracted*100
############### make CCLE and proteome  comparable ############
str(CCLE_proteome)
Row_mean<-rowMeans(CCLE_proteome)
Row_mean_extracted<-(CCLE_proteome/Row_mean)
multiplication<-Row_mean_extracted*100

#d<-length(dimnames(CCLE_proteome)[[1]])

#corr<-c(rep(0,d))
#for(i in 1:d){

 #corr[i]<-cor(proteome_NANreduced_CCLE[i,],multiplication[i,],method="spearman")

 #} 
 #names(corr)<-dimnames(proteome_NANreduced_CCLE)[[1]]
 
 str(proteome_NANreduced_CCLE)
 str(multiplication)
 
 Transcriptome<-multiplication
 Proteome<-proteome_NANreduced_CCLE
 
 
 str(Transcriptome)
 str(Proteome)
 
 Proteome<-as.data.frame(Proteome)
 Transcriptome<-as.data.frame(Transcriptome)
 
 
 x <- list(Proteome, Transcriptome)
 names(x)<-c("Proteome","Transcriptome")
sapply(x, dim)



layout(matrix(1:4, 1, 4))
par(mar=c(2, 1, 0.1, 6))
for (df in x) {
d <- dist(t(df))
 hcl <- hclust(d)
 dend <- as.dendrogram(hcl)
 plot(dend, horiz=TRUE)
 }
 
 library(omicade4)
 mcoin <- mcia(x, cia.nf=10)
 class(mcoin)
 
cancer_type <- colnames(x$Proteome)
cancer_type <- sapply(strsplit(cancer_type, split="\\."), function(x) x[1])
cancer_type
 plot(mcoin, axes=1:2, phenovec=cancer_type, sample.lab=FALSE, df.color=1:4)
 


#################################### check again only for the highly correlated genes and proteins how is the plot #########
no i can see how is the same srcture of the same time for the mots interesting vision 

d<-length(dimnames(Proteome)[[1]])

corr<-c(rep(0,d))
for(i in 1:d){

 corr[i]<-cor(Proteome[i,],Transcriptome[i,],method="spearman")

 } 
 names(corr)<-dimnames(Proteome)[[1]]

 Sort_corr<-sort(corr, decreasing = TRUE)

library(ggplot2)
Correlation<- Sort_corr
index <- Sort_corr
Gene<-c(1:length(corr))

qplot(Gene,Correlation, colour=index) + scale_colour_gradient(limits=c(-0.5, 1),low="blue", high="red")+ggtitle("Gene / Protein Spearman Correlation")+theme(plot.title = element_text(hjust = 0.5))+ ylim(-0.5,1)

high_corr<- which(corr<=0.3)

high_corr_prot<-Proteome[high_corr,]
high_corr_trans<-Transcriptome[high_corr,]

library(r.jive)
 Data_test_high_corr<- list(high_corr_prot,high_corr_trans)
 names(Data_test_high_corr)<-c("Proteome","Transcriptome")
 Results_high = jive(Data_test_high_corr)
 
 str(Results_high)
 
#heatmap((Results$joint)[[1]])
#heatmap(Results$data$Proteome)


################## visualization ################### 
png("VarExplained.png",height=300,width=450)  
showVarExplained( Results_high)  
dev.off()


png("HeatmapsBRCA.png",height=465,width=705)
showHeatmaps( Results_high, order_by=0)  #### the matrices are not re-ordered  If  order_by=0, orderings are determined by joint structure
                  





